\chapter{Прилог 1 - Алатки}
\label{ch:alatki}

Во продолжение е даден краток опис на алатките кои се искористив за изработка на трудот.

\section{Keras}

Keras е библиотека за работа со длабоки невронски мрежи на високо ниво. Напишана е во Python програмскиот јазик и дава можност за работа со користење на една од трите најпопуларни библиотеки за машинско учење на ниско ниво: TensorFlow, CNTK или Theano. Приоритет на библиотеката е овозможување на брзо и едноставно креирање на прототипи на системи со машинско учење, и е едноставна за разбирање и користење. Паралелно процесирачките способности може да се користат на процесор или графички акцелератор. Библиотеката е екстремно модуларна, скоро сите елементи во невронските мрежи се конфигурабилни модули и лесно може да се менуваат. Истите модули се отворени и може да бидат проширени или заменети од корисничка страна.

\section{Music21}

Music21 е Python библиотека за музикологија, развиена на MIT. Наменета е да овозможни лесна и детална анализа на музички композиции во симболична форма. Покрај повеќе видови музички анализи, овозможува и модифицирање на музички датотеки, како и програматско креирање на музика и музички примероци. Поддржува низа на формати, од кои најзначајни се MIDI и musicxml. Покрај тоа содржи и повеќе колекции на класична музика од јавен домен кои може да се искористат за анализа или за тренирање на модели за генерирање на музика. 

\section{pypianoroll}

Pypianoroll е Python библиотека за работа со пијано лента записи. Подржува работа со записи со една или повеќе ленти, како и можности за транспозиција на постоечки пијано ленти. Овозможува и конверзија на MIDI датотеки во пијано лента и обратно, како и графичка визуелизација на пијано лентите.